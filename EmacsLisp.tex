
%!Tex Program = xelatex


\documentclass[hyperref,UTF8,a4paper,openany]{ctexbook}
\usepackage{zhnumber}
%\usepackage{ctex}
\usepackage{graphicx, subfigure}

\usepackage{latexsym, amsmath, amssymb, amsfonts, mathrsfs, amsthm}
\usepackage{siunitx}

\usepackage{cases}

\usepackage{multirow}
\usepackage{textcomp}
\usepackage[top=2.54cm, bottom=2.54cm, left=3.18cm, right=3.18cm]{geometry}
\usepackage[toc,lof]{multitoc}

\usepackage{asymptote}

\usepackage{hyperref}
\hypersetup{colorlinks,
	linkcolor=black,
	filecolor=black,
	urlcolor=blue,
	citecolor=black,
	pdftitle={Emacs Lisp 编程语言简介},
	pdfauthor={徐策},
	pdfsubject={Emacs Lisp, 编程语言},
	pdfkeywords={Emacs, 编程语言},
	pdfproducer={XeLaTeX},
	pdfborder=0 0 1}

\setcounter{secnumdepth}{3} %使subsubsection也有编号

\usepackage{fancyhdr}	%页眉页脚

\usepackage[inline]{enumitem}
\setlist[enumerate,1]{leftmargin=0pt,itemindent=2em,itemsep=0ex,listparindent=2em,parsep = 0ex,topsep = 1ex}
\setlist[enumerate,2]{leftmargin=1em,itemsep=0ex,itemindent=2em,listparindent=2em,parsep = 0ex,topsep = 1ex}
\setlist[enumerate,3]{leftmargin=1em,itemsep=0ex,itemindent=2em,listparindent=2em,parsep = 0ex,topsep = 1ex}
\setlist[itemize,1]{itemsep = 0ex, parsep = 0ex, topsep = 1ex}
\setlist[description]{listparindent=2em, itemsep=0ex, parsep = 0ex,topsep = 1ex}

\ctexset{
	figurename={\kaishu 图},
	tablename={表},
	contentsname={目\quad 录},
	listfigurename={插图目录}
}

\newcommand{\ds}{\displaystyle}
\newcommand{\eps}{\varepsilon}
\newcommand{\mbf}{\boldsymbol}
\newcommand{\mrm}[1]{\boldsymbol{\mathnormal{#1}}}
\newcommand{\bnb}{\mbf{\nabla}}
\newcommand{\pl}{\partial}
\newcommand{\arcosh}{\mathrm{arcosh}\,}
\newcommand{\sn}{\mathrm{sn}\,}
\newcommand{\cn}{\mathrm{cn}\,}
\newcommand{\tn}{\mathrm{tn}\,}
\newcommand{\dn}{\mathrm{dn}\,}
\newcommand{\am}{\mathrm{am}\,}
\newcommand{\mathd}{\mathrm{d}}
\newcommand{\mathe}{\mathrm{e}}
\newcommand{\mathi}{\mathrm{i}}
\newcommand{\mathj}{\mathrm{j}}
\newcommand{\mathk}{\mathrm{k}}
\renewcommand{\vec}{\overrightarrow}
\renewcommand{\bf}{\heiti}
\renewcommand{\bfseries}{\heiti}
\newcommand{\ddt}{\dfrac{\mathd}{\mathd t}}
\newcommand{\difOp}[1]{\dfrac{\mathd}{\mathd #1}}
\newcommand{\dif}[2]{\dfrac{\mathd #1}{\mathd #2}}
\newcommand{\pldif}[2]{\dfrac{\pl #1}{\pl #2}}
\newcommand{\trans}{\mathrm{T}}
\renewcommand{\Re}{\mathrm{Re}\,}
\renewcommand{\Im}{\mathrm{Im}\,}

\newtheoremstyle{theoremwithoutdot}% 类型名
  {}%                   Space above, empty = `usual value'
  {}%                   Space below
  {\kaishu}%                   Body font
  {}%         Indent amount (empty = no indent, \parindent = para indent)
  {\heiti}%          Thm head font
  {}%                   Punctuation after thm head
  {1em}%                Space after thm head
  {\thmname{#1}\thmnumber{~#2}\thmnote{~(#3)}}%                   Thm head spec
  
 \newtheoremstyle{solutionstyle}% 类型名
  {}%                   Space above, empty = `usual value'
  {}%                   Space below
  {}%                   Body font
  {}%         Indent amount (empty = no indent, \parindent = para indent)
  {\heiti}%          Thm head font
  {}%                   Punctuation after thm head
  {1em}%                Space after thm head
  {\thmname{#1}\thmnumber{~#2}\thmnote{~(#3)}}%                   Thm head spec

\theoremstyle{theoremwithoutdot}
\newtheorem{defi}{定义}[section]
\newtheorem{theorem}{定理}[section]
\newtheorem{lemma}[theorem]{引理}
\newtheorem{prop}[theorem]{命题}
\newtheorem{coro}{推论}[theorem]
\newtheorem{remark}{注}
\newtheorem{example}{例}[chapter]
\newtheorem{question}{题}[section]
\newtheorem{property}[theorem]{性质}
%\renewcommand{\thequestion}{\arabic{chapter}.\arabic{section}.\arabic{question}}
%\theoremstyle{solutionstyle}
\newtheorem*{solution}{解}
%去掉证明后面的点
\makeatletter
\renewenvironment{proof}[1][\proofname]{\par%
\pushQED{\qed}%
\normalfont \topsep6\p@\@plus6\p@\relax%
\trivlist%
\item[\hskip\labelsep%
#1]\ignorespaces%
}{%
\popQED\endtrivlist\@endpefalse%
}
\makeatother
\renewcommand{\proofname}{{\heiti 证明}}

%\renewcommand{\thefootnote}{\fnsymbol{footnote}}
%带圈脚注
\usepackage{pifont}
\usepackage[perpage,stable]{footmisc}  %每页脚注重新编号
\renewcommand{\thefootnote}{\ding{\numexpr191+\value{footnote}}}
% 脚注中的脚注序号不用上标,正文中的脚注号保持不变
\makeatletter
\def\my@makefnmark{\hbox{\normalfont\@thefnmark\space}}
\let\my@save@makefntext\@makefntext
\long\def\@makefntext#1{{%
  \let\@makefnmark\my@makefnmark
  \my@save@makefntext{#1}}}

\allowdisplaybreaks[3]

\title{Emacs Lisp 编程语言简介}
\author{翻译:徐策\thanks{翻译于"An Introdaction to Programming in Emacs Lisp"(Revised Third Edition) by {Robert J. Chaeeell}}\\ Email:\url{201716021013@cqu.edu.cn}}
%\date{}

\makeatletter
\def\cleardoublepage{\clearpage\if@twoside \ifodd\c@page\else
\hbox{}
\vspace*{\fill}

\vspace{\fill}
\thispagestyle{empty}
\newpage
\if@twocolumn\hbox{}\newpage\fi\fi\fi}
\makeatother

%在插图目录图编号前加上“图”
%\usepackage{titletoc}
%\titlecontents{figure}[0.5cm]{\songti}{\figurename~\thecontentslabel\quad}{\hspace*{-1.5cm}}{\titlerule*[0.12cm]{.}\contentspage}[\addvspace{6pt}]

%样式定义
\usepackage{titlesec}
\usepackage{xcolor,colortbl}
\usepackage{tgpagella}
\usepackage[T1]{fontenc}
 
\definecolor{titlecolor}{RGB}{129,129,188}
\definecolor{contentcolor}{RGB}{129,188,129}
\definecolor{backcolor}{RGB}{129,188,129}
\newcommand{\hwyk}{\heiti}
 
\newcommand{\mytitle}[1]{
\begin{tabular}{p{0.01\textwidth}p{0.99\textwidth}}
\cellcolor{black} &\cellcolor{titlecolor} \textcolor{white}
{\newline\hwyk\LARGE 第\zhnumber{\thechapter}章 \  \ #1}
\end{tabular}
\arrayrulewidth=0.4pt
}
 
\newcommand{\mycont}[1]{
\vspace{-0.5cm}
\begin{tabular}{p{0.01\textwidth}p{0.99\textwidth}}
\cellcolor{black} &\cellcolor{contentcolor} \textcolor{white}
{\newline\hwyk\LARGE #1}
\end{tabular}
\arrayrulewidth=0.4pt
\vspace{-1.5cm}
}
 
\newcommand{\mysection}[1]{
\setlength\arrayrulewidth{1pt}\arrayrulecolor{titlecolor}
\begin{tabular}{p{0.01\textwidth}p{0.99\textwidth}}
\hline
\cellcolor{titlecolor} &  \textcolor{black}{\hwyk\LARGE \thesection ~ #1}
\end{tabular}
\arrayrulewidth=0.4pt
\vspace{-0.7cm}
}

\newcommand{\mysubsection}[1]{
\setlength\arrayrulewidth{1pt}\arrayrulecolor{titlecolor}
\begin{tabular}{p{0.01\textwidth}p{0.99\textwidth}}
\hline
\cellcolor{titlecolor} &  \textcolor{black}{\hwyk\Large \thesubsection ~ #1} \\
\hline
\end{tabular}
\arrayrulewidth=0.4pt
\vspace{-0.5cm}
}

\newcommand{\mysubsubsection}[1]{
\setlength\arrayrulewidth{1pt}\arrayrulecolor{titlecolor}
\begin{tabular}{p{0.01\textwidth}p{0.95\textwidth}p{0.01\textwidth}}
\hline
\cellcolor{titlecolor} &  \textcolor{black}{\hwyk\large \thesubsubsection ~ #1} & \cellcolor{titlecolor} \\
\hline
\end{tabular}
\arrayrulewidth=0.4pt
\vspace{-0.5cm}
}


\newcommand{\myappendix}[1]{
\begin{tabular}{p{0.01\textwidth}p{0.99\textwidth}}
\cellcolor{black} &\cellcolor{backcolor} \textcolor{white}
{\newline\hwyk\LARGE 附录\thechapter \  \ #1}%\Alph{\thechapter}
\end{tabular}
\arrayrulewidth=0.4pt
}

 
\newcommand{\appendixsection}[1]{
\setlength\arrayrulewidth{1pt}\arrayrulecolor{backcolor}
\begin{tabular}{p{0.01\textwidth}p{0.99\textwidth}}
\hline
\cellcolor{backcolor} &  \textcolor{black}{\hwyk\LARGE \thesection ~ #1}
\end{tabular}
\arrayrulewidth=0.4pt
\vspace{-0.7cm}
}

\newcommand{\appendixsubsection}[1]{
\setlength\arrayrulewidth{1pt}\arrayrulecolor{backcolor}
\begin{tabular}{p{0.01\textwidth}p{0.99\textwidth}}
\hline
\cellcolor{backcolor} &  \textcolor{black}{\hwyk\Large \thesubsection ~ #1} \\
\hline
\end{tabular}
\arrayrulewidth=0.4pt
\vspace{-0.5cm}
}

\newcommand{\appendixsubsubsection}[1]{
\setlength\arrayrulewidth{1pt}\arrayrulecolor{backcolor}
\begin{tabular}{p{0.01\textwidth}p{0.95\textwidth}p{0.01\textwidth}}
\hline
\cellcolor{backcolor} &  \textcolor{black}{\hwyk\large \thesubsubsection ~ #1} & \cellcolor{backcolor} \\
\hline
\end{tabular}
\arrayrulewidth=0.4pt
\vspace{-0.5cm}
}

\newcommand{\mybackmatter}[1]{
\begin{tabular}{p{0.01\textwidth}p{0.99\textwidth}}
\cellcolor{black} &\cellcolor{backcolor} \textcolor{white}
{\newline\hwyk\LARGE #1}
\end{tabular}
\arrayrulewidth=0.4pt
}

\begin{document}

\begin{asydef}
	texpreamble("\usepackage{xeCJK}");
	texpreamble("\setCJKmainfont{SimSun}");
	usepackage("amsmath");
	usepackage("amssymb");
	usepackage("siunitx");
	import graph;
	import math;
\end{asydef}

\maketitle

\frontmatter

\pagestyle{fancy}
\renewcommand{\chaptermark}[1]{\markboth{第\zhnumber{\thechapter}章\ \ #1}{}}
\renewcommand{\sectionmark}[1]{\markright{\thesection\ \ \rm #1}{}}	%这两个命令第一次必须出现在\pagestyle{fancy}之后,否则\pagestyle{fancy}会覆盖掉其效果
\markboth{\leftmark}{\rightmark}
\fancyhf{}
\fancyhead[CO]{\rightmark}
\fancyhead[LE,RO]{$\cdot$\, \thepage\, $\cdot$}
\fancyhead[CE]{\leftmark}
\renewcommand{\headrulewidth}{0.4pt}

\titleformat{\chapter}[hang]{\hwyk\LARGE}
{}{0mm}{\hspace{-0.4cm}\mycont}

\tableofcontents

\cleardoublepage

\mainmatter
 
\titleformat{\chapter}[hang]{\hwyk\LARGE}
{}{0mm}{\hspace{-0.4cm}\mytitle}
 
\titleformat{\section}[hang]{\hwyk\LARGE}
{}{0mm}{\hspace{-0.5cm}\mysection}

\titleformat{\subsection}[hang]{\hwyk\large}
{}{0mm}{\hspace{-0.5cm}\mysubsection}

\titleformat{\subsubsection}[hang]{\hwyk\large}
{}{0mm}{\hspace{-0.5cm}\mysubsubsection}
 
\titlespacing{\chapter}
{0pc}{1.5ex plus .1ex minus .2ex}{.2pc}


\setcounter{chapter}{-1}
\chapter{前言}
大多数的$GNU Emacs$集成环境都是用编程语言$Emacs Lisp$写成。用这种编程语言写成的软件——一系列指令——当你给出命令时,告诉电脑要做什么。Emacs被设计用来作为一种便于安装,对编辑器的扩展,来编写新的代码。\par 
($GNU Emacs$有时被称作“可扩展编辑器”,但它的功能远不止编辑文本,将$Emacs$称为“可扩展计算环境”会更好。然而,那个短语有点太口语化。将$Emacs$直接称为编辑器要简单许多。此外,你在$Emacs$上做的每一件事——发现玛雅日期或者月相,简化多项式,测试代码,管理文件,阅读书信,写书——所有这些活动都是用最一般意义上字词做各种各样的编辑。)\par
尽管$Emacs Lisp$




\section{经典力学的发展}

17世纪由Galileo和Newton奠基。1687年发表的《自然哲学之数学原理》标志着经典力学体系的建立,即Newton力学。
\begin{itemize}
    \item {\heiti 核心概念}:质点、力、加速度;
    \item {\heiti 基本原理}:Newton三大定律;
    \item {\heiti 数学方法}:三维实空间几何、矢量代数。
\end{itemize}

18至19世纪,d'Alembert、Euler、Lagrange、Hamilton、Jacobi、Gauss、Poisson等建立力学的后Newton形式——分析力学。
\begin{itemize}
    \item {\heiti 核心概念}:能量、作用量;
    \item {\heiti 基本原理}:Hamilton原理(最小作用量原理);
    \item {\heiti 数学方法}:抽象空间、数学分析;
    \item {\heiti Lagrange形式}:广义坐标的引入使力学具备描述其它非力学体系的潜力。
    \item {\heiti Hamilton形式或正则形式}:力学体系内在对称性的揭示开始支配物理学思维,从而完成从力学到相对论和量子力学的升华。
\end{itemize}

分析力学给出了动量、角动量和能量的普适定义,使力学成为整个物理学的原型(量子力学、统计物理和量子场论的理论基石),为物理理论的统一奠定基础。

分析力学当代的发展结果:非线性物理与混沌、随机性、宿命论因果律破产。

\section{经典力学的适用范围}

{\heiti 宏观物体}在{\heiti 弱引力场}中的{\heiti 低速}运动。

\begin{itemize}
    \item {\heiti 高速运动}:狭义相对论
    \item {\heiti 强引力场}:广义相对论
    \item {\heiti 微观体系}:量子力学
\end{itemize}

% \section{课程的价值取向}

% 经典力学:通向现代物理宏伟圣殿的阶梯,非线性前沿科学成长的摇篮。

% 总结先贤理论遗产精华,领悟物理学的基本原理、通用语言和理论方法。%——醉翁之意不在酒:着眼于整个物理学。

% \begin{quote}
    % \kai ……科学知识使人们能制造许多产品,做许多事业。……科学的另一个价值是提供智能与思辩的享受。……如果我们社会进步的最终目标正是为了让各种人能享受它想做的事,那么科学家们思辩求知的享受就和其它事具有同等的重要性了。
%   \begin{flushright}
        % —— R.P. Feynman,《科学的价值》
    % \end{flushright}
% \end{quote}%第零章——前言

\input{chapter/chapter1}%第一章——List 处理

\input{chapter/chapter2}%第二章——实现编译

\input{chapter/chapter4}%第三章——如何定义函数

\input{chapter/chapter5}%第四章——几个与缓冲区有关的函数

\input{chapter/chapter6}%第五章——几个复杂一点的函数

\input{chapter/chapter8}%第六章——变窄与拓宽

\input{chapter/chapter7}%第七章——car,cdr,cons:基本函数

\input{chapter/chapter3}%第八章——剪切与存储文本

\input{chapter/chapter9}%第九章——List是如何运行

\input{chapter/chapter10}%第十章——召回文本

\chapter{循环与递归}%第十一章——循环与递归

\chapter{常规表达式查找}%第十二章——常规表达式查找

\chapter{用重复与正则表达式计数}%第十三章——用重复与正则表达式计数

\chapter{函数定义中的字数统计}%第十四章——函数定义中的字数统计

\chapter{准备图片}%第十五章——准备图片

\chapter{你的.emacs文件}%第十六章——你的.emacs文件

\input{chapter/chapter17}%第十七章——调试

\input{chapter/chapter18}%第十八章——总结


%以下插入附录


\appendix

\titleformat{\chapter}[hang]{\hwyk\LARGE}
{}{0mm}{\hspace{-0.4cm}\myappendix}
 
\titleformat{\section}[hang]{\hwyk\LARGE}
{}{0mm}{\hspace{-0.5cm}\appendixsection}

\titleformat{\subsection}[hang]{\hwyk\large}
{}{0mm}{\hspace{-0.5cm}\appendixsubsection}

\titleformat{\subsubsection}[hang]{\hwyk\large}
{}{0mm}{\hspace{-0.5cm}\appendixsubsubsection}

\renewcommand{\chaptermark}[1]{\markboth{附录\ \ #1}{}}
\renewcommand{\sectionmark}[1]{\markright{\thesection\ \ \rm #1}{}}	%这两个命令第一次必须出现在\pagestyle{fancy}之后,否则\pagestyle{fancy}会覆盖掉其效果
\markboth{\leftmark}{\rightmark}

\input{chapter/appendixA.tex}%附录——the-the函数

\input{chapter/appendixB.tex}%附录——运用the Kill Ring

\input{chapter/appendixC.tex}%附录——含标记轴的图片

\input{chapter/appendixD.tex}%附录——自由软件和免费手册

\input{chapter/appendixE.tex}%附录——GNU 免费文档许可


%%索引页


\backmatter
 
\titleformat{\chapter}[hang]{\hwyk\LARGE}
{}{0mm}{\hspace{-0.4cm}\mybackmatter}

\addcontentsline{toc}{chapter}{\listfigurename}{%
\let\oldnumberline\numberline%
\renewcommand{\numberline}{\figurename~\oldnumberline}%
\listoffigures
}

%参考文献
\input{chapter/bibliography.tex}

\end{document}
